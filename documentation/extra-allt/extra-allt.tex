\documentclass[10pt,english,a4paper
	%,twoside
]{article}

% The following is not required -- igore it if you don't have it.
\usepackage{pax3}

\usepackage{amsmath,hyperref,makeidx}


\newcommand{\lidar}{Lidar}
\newcommand{\slu}{\textsc{slu}}
\newcommand{\download}{“Skogsdatalabbet” (\url{where.to.download})}
\newcommand{\tabrow}[1]{\mname{#1\_1ret}	& \mname{#1\_1ret\_ge150cm}	& \mname{#1\_all}	& \mname{#1\_all\_ge150cm}	\\}


\title{Swedish {\lidar} Metrics}
\author{Peder Axensten, Jörgen Wallerman, and Rubén Valbuena}
\setcounter{secnumdepth}{0}


\newcommand{\mname}[1]{{\small\texttt{#1}}}


\newlength{\descriptionwidth}
\settowidth{\descriptionwidth}{\textsc{Description:~~\mbox{}}}
\newlength{\boxwidth}
\setlength{\boxwidth}{\textwidth}
\addtolength{\boxwidth}{-\descriptionwidth}

\newcommand{\metricrow}[2]{\makebox[\descriptionwidth][t]{\textsc{#1}:}\parbox[t]{\boxwidth}{#2}}
\newcommand{\metrictag}[2]{\mname{#1}\index{#1@\mname{#1} (tag)|see{#2}}}
\newcommand{\metric}[3][]{\subsubsection{#2\index{#2}}\metricrow{Tag}{\metrictag{#3}{#2}#1}}
\newcommand{\metricdescr}[1]{\\\metricrow{Description}{#1}}
\newcommand{\metriceq}[1]{{\small$$#1$$}}



\newcommand{\where}{,\quad\textrm{where }}
\newcommand{\idxsee}[2]{\index{#1|see{#2}}}
\newcommand{\aka}[2]{Also known as #1\idxsee{#1}{#2}}
\newcommand{\meanz}{\ensuremath{\bar{z}}}

\renewcommand{\indexspace}{}
\renewcommand{\hyperindexformat}[2]{#1\mbox{,} \hyperpage{#2}}

\newcommand{\rawsum}[1]{\sum_{i=1}^n (z_i-\meanz)^#1}
\newcommand{\rawsumfrac}[2]{\frac{\rawsum{#1}}{\left[\rawsum{2}\right]^{#2}}}
\newcommand{\popsum}[1]{\frac{1}{n}\;\rawsum{#1}}
\newcommand{\samplesum}[1]{\frac{1}{n-1}\;\rawsum{#1}}




\makeindex
\begin{document}
\maketitle


\section{Introduction}\label{introduction}

To produce the Swedish forest variables (published via the Swedish Forest Agency) we use models with specific metrics as independent data and data from the Swedish \textsc{nfi} as dependent data. Thus, we calculate metrics in raster form as well as for the \textsc{nfi} plots. As such metrics may be useful also in other situations, we make a set of metrics available through {\download}. 

This document contains information on the available metrics. 


\subsection{What the metrics may be used for}

The metrics are published as open data under the Simplified \textsc{bsd} License\footnote{\url{https://en.wikipedia.org/wiki/BSD_licenses}} and may be used for any non-criminal purpose as long as the following is clearly stated:
\emph{“Data is produced by the Swedish University of Agriculture (\textsc{slu}), for more information see \url{www.slu.se}.”}


\subsection{What the metrics can be used for}

We use a subset of the metrics to produce the Swedish forest variables published through the Swedish Forest Agency.\footnote{\url{https://skogsstyrelsen.se/skogligagrunddata} (in Swedish)}


\subsection{Files}

To define a metric you combine a statistic and a filter, the naming convention is year\_statistic\_filter as detailed in “\nameref{sec:statistics}”, “\nameref{sec:details}”, and “\nameref{sec:filtering}”, below.

\subsubsection{Rasters}
All the available metrics are saved as geotiff raster files (suffix \mname{.tif}) using lossless compression and the \textsc{sweref}99 \textsc{tm} projected georeference system. The resolution is 10$\times$10~m$^2$.

% \subsubsection{Text file}
% Along with the raster files, there is a text file \mname{year\_plots.csv} containing the metrics as well as forest variables from the Swedish \textsc{nfi} permanent plots. The text file uses semicolon as column separator. The plot coordinates are not included as they are not published by the Swedish \textsc{nfi}.


\section{Metrics}\label{sec:metrics}

\subsubsection{Available statistics}\label{sec:statistics}
\begin{itemize}
	\item The number of points: \mname{count}. You probably do not want to use \mname{count} directly, but rather divide two counts with different filtering to get a proportion. 
	\item Summary statistics: \mname{mean}, \mname{mean2}, \mname{variance}, \mname{skewness}, and \mname{kurtosis}.
	\item Percentiles: \mname{p10}, \mname{p20}, \mname{p30}, \mname{p40}, \mname{p50}, \mname{p60}, \mname{p70}, 
								\mname{p80}, \mname{p90}, and \mname{p95}. 
	\item L-moments: \mname{L2} (\textsc{l}-scale), \mname{L3} (\textsc{l}-skewness), and \mname{L4} (\textsc{l}-kurtosis). 
	\item Median absolute deviation: \mname{mad}. 
\end{itemize}

\subsubsection{Filtering}\label{sec:filtering}
You choose if you want all points (\mname{all}) or only the first returns (\mname{1ret}). Optionally you also choose a boundary (\mname{\_geHcm}).
\begin{itemize}
	\item \mname{all} (all points) or \mname{all\_geHcm} (all points greater than or equal to $H$ cm). 
	\item \mname{1ret} (all first returns) or \mname{1ret\_geHcm} (all first returns greater than or equal to $H$ cm).
\end{itemize}
All the statistics come with at least the following filters (\mname{count} has a few extra):
\\{\small\begin{tabular}{@{}l@{, }l@{, }l@{, and }l@{.}}
	\tabrow{\mbox{}}
\end{tabular}}


\subsubsection{Examples}
\begin{itemize}
	\item \mname{2022\_variance\_all} (sample variance of all points), 
	\item \mname{2022\_variance\_1ret} (sample variance of first returns), and 
	\item \mname{2022\_count\_1ret\_ge150cm} (number of first returns greater than or equal to 1.5~m). 
\end{itemize}


\subsection{Definitions}\label{sec:details}

\begin{itemize}
	\item $n$, the number of values.
	\item $z_i$, value $i$, where $1\le i\le n$.
	\item $z_{(i)}$, ordered value $i$, where $1\le i\le n$ and $z_{(i)}\le z_{(i+1)}$.
	\item $\meanz$, the mean.
\end{itemize}

\metric{Number of points}{count}
\metricdescr{The number of $z$-values.}

\metric[ and \metrictag{mean2}{xxxx}]{Mean value}{mean}
\metricdescr{The mean and the mean of the squares, respectively.}
\metriceq{\frac{1}{n}\;\sum_{i=1}^n z_i\quad\textrm{ and }\quad\frac{1}{n}\;\sum_{i=1}^n z_i^2}

%\metric[ and \metrictag{sample\_stddev}{Standard deviation}]{Standard deviation}{pop\_stddev}
\metric{Standard deviation}{–}%{stddev}
\metricdescr{The standard deviation is the same as the square root of the variance and is not included – use $\sqrt{\textrm{\mname{variance}}}$ instead.}
%\metriceq{\sqrt{\popsum{2}}\quad\textrm{(population standard deviation)}}
%\metriceq{\sqrt{\samplesum{2}}%\quad\textrm{(sample standard deviation)}
%}

%\metric[ and \metrictag{sample\_variance}{Variance}]{Variance}{pop\_variance}
\metric{Variance}{variance}
\metricdescr{The sample variance.}
% \metricdescr{The population and sample variance. Presently, no variance is distributed as the square of the standard variation can be used.}
% \metriceq{\popsum{2}\quad\textrm{(population variance)}}
\metriceq{\samplesum{2}%\quad\textrm{(sample variance)}
}

%\metric[ and \metrictag{sample\_skewness}{Skewness}]{Skewness}{pop\_skewness}
\metric{Skewness}{skewness}
\metricdescr{The sample ($G_1$)\index{G1 skewness@$G_1$ skewness|see{Skewness}} skewness.}
%\metricdescr{The population ($g_1$)\index{g1 skewness@$g_1$ skewness|see{Skewness}} and sample ($G_1$)\index{G1 skewness@$G_1$ skewness|see{Skewness}} skewness. Presently, only the sample skewness is distributed.}
%\metriceq{\sqrt{n}\cdot\rawsumfrac{3}{1.5}\quad\textrm{(population skewness)}}
\metriceq{\frac{\sqrt{n(n-1)}}{n-2}\cdot\sqrt{n}\cdot\rawsumfrac{3}{1.5}%\quad\textrm{(sample skewness)}
}



% Same as skewness_G1( s_ )
% Same as skewness_g1( s_ )*fisher_pearson_coeff( s_ )
% Same as cmoment_raw< 3 >( s_ )/( m*std::sqrt( m/n ) )*fisher_pearson_coeff( s_ )
% where m is cmoment_raw< 2 >( s_ )
% and fisher_pearson_coeff( s_ ) is std::sqrt( n*(n-1) )/( n-2 )

%\metric[ and \metrictag{sample\_kurtosis}{Kurtosis}]{Kurtosis}{pop\_kurtosis}
\metric{Kurtosis}{kurtosis}
\metricdescr{The sample kurtosis.}
%\metricdescr{The population and sample kurtosis. Presently, only the sample kurtosis is distributed.}
%\metriceq{n\cdot\rawsumfrac{4}{2} - 3\quad\textrm{(population kurtosis)}}
% Same as population_kurtosis_fisher( s_ )
% Same as cmoment_raw< 4 >( s_ )*n/( m2*m2 ) - 3
% where m2 is cmoment_raw< 2 >( s_ )
% where cmoment_raw< p >( s_ ) is \sum_i (x_i-\bar{x})^p
\metriceq{\frac{(n+1)(n-1)}{(n-2)(n-3)}\cdot n\cdot\rawsumfrac{4}{2} - \frac{3n^2}{(n-2)(n-3)}%\quad\textrm{(sample kurtosis)}
}
% Same as sample_kurtosis_fisher( s_ )
% Same as sample_kurtosis_pearson( s_ ) - 3.0f*n*n/( (n-2)*(n-3) )
% Same as cmoment_raw< 4 >( s_ )*n*( n+1 )/( ( n-1 )*( n-2 )*( n-3 )*v*v ) - 3.0f*n*n/( (n-2)*(n-3) )
% Same as ( cmoment_raw< 4 >( s_ ) / v^2 )*n*( n+1 )/( ( n-1 )*( n-2 )*( n-3 ) ) - 3.0*n*n/( (n-2)*(n-3) )
% where v is sample_variance( s_ )

\metric{Percentile $k$}{p{\it k}}
\metricdescr{The value $z_{(p)}$ for which $k$~\% of all values are smaller: $p=\frac{k}{100}\cdot n$. When $p$ is not an integer value, $z_{(p)}$ is the linear interpolation of the two adjacent values.
	\idxsee{Median}{Percentile with $k$ = 50}
	\idxsee{Quartile $q$}{Percentile with $k$ = 25$q$}
	\idxsee{Quantile $q$}{Percentile with $k$ = 100$q$}}

\metric{Median absolute deviation}{mad}
\metricdescr{Median absolute deviation: $\textrm{median}(|z_i - \meanz|)$}

\newcommand{\Lmoment}[2]{\frac{1}{#1}\binom{n}{#1}^{-1}\;\sum_{i=1}^n \left[#2\right]\cdot z_{(i)}}

\metric{L-mean}{–}%{L1}
\metricdescr{The mean has the same value in ordered statistics as in unordered, so \mname{L1} is not included – use \mname{mean} instead.}
%\metriceq{\frac{1}{n}\;\sum_{i=1}^n z_{(i)} = \meanz}

\metric{L-scale}{L2}
\metriceq{\Lmoment{2}{\binom{i-1}{1} - \binom{n-i}{1}}}

\metric{L-scewness}{L3}
\metriceq{\Lmoment{3}{\binom{i-1}{2} -2\binom{i-1}{1}\binom{n-i}{1} + \binom{n-i}{2}}}

\metric{L-kurtosis}{L4}
\metriceq{\Lmoment{4}{\binom{i-1}{3} - 3\binom{i-1}{2}\binom{n-i}{1} + 3\binom{i-1}{1}\binom{n-i}{2} - \binom{n-i}{3}}}

% \metric{L-scewness ratio}{L3\_ratio}
% \metricdescr{\aka{L3-moment ratio}{L-scewness ratio}.\\L-moment ratios lie within the interval (–1, 1). Tighter bounds can be found for some specific L-moment ratios. Presently, the L-scewness ratio is not distributed as it is easily calculated from L-scewness and L-scale.}
% \metriceq{\frac{\textrm{L-scewness}}{\textrm{L-scale}} = \frac{\texttt{L3}}{\texttt{L2}}}
%
% \metric{L-kurtosis ratio}{L4\_ratio}
% \metricdescr{\aka{L4-moment ratio}{L-kurtosis ratio}.\\L-moment ratios lie within the interval (–1, 1). Tighter bounds can be found for some specific L-moment ratios. Presently, the L-kurtosis ratio is not distributed as it is easily calculated from L-kurtosis and L-scale.}
% \metriceq{\frac{\textrm{L-kurtosis}}{\textrm{L-scale}} = \frac{\texttt{L4}}{\texttt{L2}}}

\hspace{2em}


\section{List of all the metrics}

{\small
\begin{tabular}{@{}llll@{}}
	\hline
	\tabrow{count}
	\tabrow{kurtosis}
	\tabrow{L1}
	\tabrow{L2}
	\tabrow{L3}
	\tabrow{L4}
	\tabrow{mad}
	\tabrow{mean}
	\tabrow{mean2}
	\tabrow{p10}
	\tabrow{p20}
	\tabrow{p30}
	\tabrow{p40}
	\tabrow{p50}
	\tabrow{p60}
	\tabrow{p70}
	\tabrow{p80}
	\tabrow{p90}
	\tabrow{p95}
	\tabrow{skewness}
	\tabrow{stddev}
	\hline
	count\_1ret\_ge500cm	& count\_1ret\_ge1000cm	& \multicolumn{2}{l}{count\_1ret\_ge1500cm}	\\
	count\_all\_ge500cm	& count\_all\_ge1000cm	& \multicolumn{2}{l}{count\_all\_ge1500cm}	\\
	\hline
\end{tabular}}

%{\small\printindex}		% Run "makeindex public-metrics.idx" from that directory.
\end{document}
